\documentclass{article}
\usepackage[utf8]{inputenc}

\title{Go Report}
\date{January 2019}

\usepackage{natbib}
\usepackage{graphicx}
\usepackage{graphics}


\begin{document}
\author{Sarah Roediger - NU Lichting 4} 
\maketitle

\section{Design Choices}

\subsection{Hierarchy - Model Classes}

\subsubsection{Board}

The board does only keep track of the current board state. It does not enforce rules or validates any moves.

\subsubsection{Game}

All rules should be enforced in the game. The game uses the board to keep track of the game process. In this way, in a later state the rules can also easily be changed when another game might be played.



\r
Rules the game should enforce
\begin{enumerate} 

	\item Pass - Stopping the game when both players passed
	\item Black makes the first move
	\item Configurable size (could also be done in the board, but we want the board to not worry about sizes)
	\item Checks whether move would recreate past boardState
	\item 

\end{enumerate}

\subsubsection{Player}

The player knows when a move generally is valid (isField and isValid), since these rules are unlikely to change, but whether the move would recreate a previous boardState is checked in the game, since this rule could be changed in future games.

 does not know anything about the current gameState?



\section{Things to still keep in mind}

\begin{enumerate}

	\item Exceptions 
	\item tests
	\item hierarchieen

\end{enumerate}



\end{document}
